\renewcommand\thesection{\arabic{section}}
\section{Project Area Risks}    

\subsection{Student Contribution}
\subsubsection{A team member not contributing or engaging enough}

An example of a possible risk is a team member not contributing and engaging enough. This could result in an unequal contribution to the tasks. It might occur by the team member missing multiple meetings or not carrying out the given tasks.
\\

There could be several root causes of this risk. It could be due to an illness, lack of time management or lack of motivation. If the risk is caused by an illness such that the team member cannot complete the task, the solution is to assign the tasks to other team members and adjust the deadlines at the weekly team meeting. If the risk is caused due to a failure of time management, we will reach out to the team member, offer support and determine why the team member couldn’t manage the workload present to them. 
\\

A solution could be prioritising the remaining tasks so the team member can focus on completing the most urgent task first. This solution could be followed by having regular meetings to keep track of the progress of the tasks. If team members fail to complete the task, the adjusted deadline and tasks could be determined at the next meeting. 
\\

If the team member's behaviour impacts the progress of the project and the team member refuses to communicate their status with the team, we will inform the teaching team of the team member’s situation and seek assistance regarding the next steps. 


\subsection{Team Self Management}
\subsubsection{Disagreements and Conflicts}
A conflict between two teammates where one feels that something must be done in a certain way but the other teammate feels that it is preferable to do it in a different way can serve as an example. If neither teammate is prepared to compromise, the dispute could turn into a conflict, which would obstruct communication and hinder teamwork, which would ultimately have an impact on the team's performance and put the project's success at risk.
\\

Poor communication and different priorities among team members are the main core causes of this risk. When two or more team members have conflicting priorities, it can be difficult to agree on what should be the team's main priority. Moreover, poor communication would result in misunderstandings and conflicts because certain teammates might not have received crucial information or may not have fully comprehended the other’s viewpoint.
\\

To minimise this risk, we must make sure that the team as a whole has a shared vision, meaning that everyone is aware of both the team's overall aims and objectives as well as their individual responsibilities. This can be done by following up on the weekly meetings to discuss everyone's contributions and preparing for the upcoming week's plan along with everyone's responsibilities for that week. In order to ensure that everyone is on the same page and contributes equally and without any misunderstandings, it will be made sure that the team meets regularly each week to discuss the required goals that must be completed by the end of the following week. These meetings can be documented in a minutes document that contains the key information of the meeting. By keeping weekly minutes available for all team members to read and view, it is possible to reduce the amount of bad communication and keep everyone informed of the team's aims and objectives.
\\

When a difference of opinion arises between two or more team members and begins to escalate into a conflict, the teaching team must be made aware of the situation. This only applies when the team cannot resolve the conflict itself and the team members are not responding to team communications.


\subsection{Teaching Staff Contribution}

\subsubsection{Sudden change of assessment dates by teaching staff}

The teaching staff responsible for our group could alter an assessment date so it is sooner/later than expected. The risk created by this change is more so when a date is shifted earlier. We often plan to allocate tasks amongst the group so that the submission is ready for the assigned deadline. Shifting this forward changes the initial plan, cutting out the time we thought we would have to work, potentially impacting the quality of our submission on the earlier date and creating unrealistic expectations.
\\

A root cause of this risk is the lack of communication between the teaching staff and us students. Informing students of a change like a submission date is vital for the group’s planning so we can adapt to the new change and not hinder the quality of our submission. This risk can be tackled by keeping communication between our group and the teaching staff high throughout the project. Not only does this keep both sides informed about what is happening, but it also helps to increase engagement and build a better relationship between the teaching staff and us students. This is important for helping everyone involved feel more comfortable, so collaborating becomes much easier to deliver an excellent final submission.
\\

A plan of action around this could involve regular weekly meetings with the teaching staff (particularly our mentor) and setting up communication channels like Email or Teams to easily reach out to one another. The plan starts by assessing the availability and allocating a time when both the team and the mentor are free. Once sorted, we will know the plan is being followed when both parties attend the meetings regularly, if not all the time. If someone cannot attend a meeting, whoever can’t attend must inform both the group and the teaching staff and can look to either reschedule or catch up later with those present at the meeting.
\\

Increasing the overall communication will help keep all involved up to date with essential information like updates to submission times. This risk should be sent to the teaching team when the shift in date makes submission completion unfeasible for our group creating unrealistic expectations. This should be raised with the teaching staff as soon as the group discovers this.


\subsubsection{Teaching staff doesn't reply or is constantly unavailable}
Teaching staff doesn’t reply to the problems raised by the students. This can be due to a personal emergency or other unforeseen circumstances  that are out of their control and cannot be dealt with swiftly. Another plausible scenario could be a lack of communication between the unavailable staff and the students or other teaching staff resulting in confusion and disorganisation due to the breakdown in communication.
\\

This can be minimised by having more than one teaching staff that may be able to communicate with and help the students that are having issues. On the students end, the risk can be minimised through constant communication as this helps keep everyone informed and prevents delays to the project.
\\

Having another teaching staff as backup for communication when the initial communicator is unavailable. If these are not in place, students can escalate the issue and receive timely help from an alternative staff member that could guide the team towards possible solutions. As soon as communication is unavailable, it is imperative that students contact the teaching team to get the issue resolved and minimise delays to the project.


\subsection{Client Contribution}
\subsubsection{Client disagreement}
As this project is based on the client's Request for Proposal, this results in considering the constraints and risks of the specifications the clients need. There could be a disagreement that could arise due to a multitude of issues, some of which are listed - the clients being dissatisfied, lapse in communication between the client and the team, inability to reach a consensus between the team and the client after multiple back and forth discussions etc. The first example could be while developing the prototype, we could run into edge cases, and to capture these, we would have to alter the initial considerations. The second example could be if an unseen change is required and the team couldn’t contact the client, the project's progress may stop or continue without the client's feedback to meet deadlines.
\\

A few root causes are the lack of communication with the client, inconsistent feedback, clash in opinions or not enough information for the clients to understand. A simple solution could be setting specific meeting times, especially if any significant changes are required. Plans could be followed by creating a detailed schedule between the clients and ensuring availability for these agreed meetings does not clash. The information should always be simplified before being passed onto the client to ensure the client can easily understand and follow along.
\\

If it comes to a situation where there is no engagement due to the availability of clients or constant meetings missed, then the teaching team shall be informed and asked regarding how to proceed forward.

\subsection{Resources}

\subsubsection{Corruption of Files}
An example of file corruption is when multiple developers are working on the same file. Developer A and B are working on the same file, and Developer A commits and pushes changes to Git while Developer B is still working on the same file. When Developer B makes changes and pushes to Git, there will be a merge conflict. If Developer B does not resolve the merge conflict or if they accidentally delete or overwrite some necessary code during the conflict resolution process, the file can become corrupted.
\\

A root cause of the risk is if the team is not informed about how we will manage our version control to prevent the corruption of files before starting development with the software. A lack of experience using version control and testing will be another cause of file corruption because if the member does not know the best practices of version control and testing, they may cause significant merge conflict, thereby increasing the chance of data corruption.
\\

Some solutions we discussed as a team include using version control Git, which has already been supplied to us via the GitHub Classroom Repository. To minimise the possible risk of file corruption, we have decided to follow the best practices such as committing frequently, resolving merge conflicts when applicable, regularly backing up the codebase and restricting merging on pull requests.
\\

Testing our implementation frequently is another way to minimise the chances of having corrupted files. This allows us to catch bugs and other issues that might cause file corruption as the codebase is growing. Another solution could be to have other people peer review the code as errors can be easily missed by the member that wrote the code.
\\

Regarding informing the teaching team, we have collectively decided as a team that if the issue is not able to be resolved in a week's time, we will contact the teaching team as soon as possible to give them ample time to respond to our issue of corruption of files.


\subsubsection{Supply Chain issues}

A supply chain issue is defined as an external force acting on a company/organisation that prevents them from manufacturing, selling or shipping products. Some root causes of the risk are listed below:-
\\

\emph{Manufacturing Shutdown}: If one of the key supply chain suppliers either shut down, goes out of business either due to financial instability or due to lack of raw materials.
\\

\emph{Transportation Delay}: If there are any delays in shipping or transportation, it could cause delays in receiving the necessary products such as resistors, capacitors, discrete ICs etc. This would also impact production schedules, delivery times, and product availability.
\\

\emph{Component becoming Obsolete}: If a manufacturing company suddenly stops production of a component/Integrated Circuit due to the supply-demand mismatch or unavailability of raw materials/components.
\\

For supply chain issues that involve either the component becoming obsolete or the manufacturing company shutting down, designing the electrical schematic with alternative components that share very similar specifications (similar price, similar technical specifications) could minimise the effect it has on the project. However, in case of unforeseen transportation delays ordering the same component from different suppliers or preferring companies that have a much shorter transportation chain would reduce the risk of transportation delays.
\\

In the case of designing with alternative parts, Altium Designer has the capability to store the alternative part number in your schematic/PCB design. Moreover, Altium designer also has the capability to create variants wherein different PCBs can be easily created depending on the availability of the part. The teaching team will be informed when there is a significant delay in the part item i.e. a week has passed, and there is no information on the arrival of the part. Additionally, the teaching team will be informed when an alternative part for the design cannot be found.